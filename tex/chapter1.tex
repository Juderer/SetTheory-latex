\chapter{集合与集合的运算}

\section{引言}

集合论成为一门学科,是十九世纪后期的事情。
G. Cantor在1874-1897发表的一系列论文奠定了集合论的基础。
从那以后,集合论的概念和结果被广泛应用于数学的各个分支,使数学学科的面貌受到了深刻的影响。
在今天,可以说集合论已成为几乎所有数学分支的基础了。

\textbf{Cantor集合论}的出现在当时数学界引起极大的反应。
它受到一部分数学家,如R. Dedekind,B. Russell,D. Hilbert等等的支持和高度赞美;
也受到一部分数学家,特别是L. Kroneeker的激烈反对。
同时,从上一世纪末开始,形形色色的有关集合论的悖论不断出现,当时的集合论对这些悖论不能做出满意的回答。

在集合论的初创时期,Cantor是以所谓“朴素的”观点来看待集合的。
他广泛地谈论集合,但他没有明确规定对于已知集合做哪些事情是合法的,就是说,对已知集合做哪些事情可以得出受到承认的集合。
尽管Cantor本人实际已经建立起相当广泛而深刻的集合理论,但他在理论基础上的某些不明确性使得他和他的支持者们不能有力地保卫他的理论,也不能把所有已经提出的悖论排除于集合论的大门之外。

为了填补Cantor在理论基础上的不足,从而维护Cantor的理论,在1908年,E. Zermelo首先为集合论设立了一套比较完整的公理,这些公理主要是明确了对已知集合做哪些事是合法的。
以后经过A. Fraenkel等人的补充和完善,形成了现在所谓的(ZF)公理系统。
较晚一些,还有所谓的(GB)公理系统,是由von Neumann,P. Bernays,K. G$\ddot{o}$del等人建立的。
在这样的公理系统中,悖论被排除了,责难的声音也就减弱了。
在二十世纪,在公理化的集合论中,关于选择公理和连续统假设的研究大大推动了集合论的发展,使之成为至今活跃的数学学科之一。

本书是为学习其他学科服务的集合论教材,不包含集合论的专门课题。
不过,按照当代的要求,本书将按照公理化的精神向读者介绍集合论的基本内容,所遵循的公理系统就是上述的(ZF)系统。
另一方面,为了便于初学者理解,本书将不追求叙述的高度形式化。
最后,我们声明,本书将承认数学中通用的逻辑法则。
除在个别场合以底注形式列出一些逻辑法则外,我们将不对逻辑法则进行系统的考察。

\section{集合}

什么是集合?集合论的创始人G. Cantor曾作如下描述:

“一个集合是我们直觉中或理智中的,确定的,互不相同的事物的一个汇集,被设想为一个整体(单元)”。

这些事物叫做这集合的元素,或说这些元素属于这集合,也说这集合包含这些元素。

Cantor的描述对于人们直观地理解集合概念是很有价值的。
例如,它说一个集合的元素是“确定的”,这意味着某个事物是否属于某个集合是没有丝毫含混余地的;
又如,它说一个集合的元素是“互不相同的”,这意味着在一个集合中相同的元素只能算是一个;
又如,它说把一个集合“设想为一个整体(单体)”,这意味着要把整个集合看成一个单独的思维对象,而不再看成那些个别元素的简单积累。

读者可以在这样的理解下举出很多集合的例子,如在某一时刻在某一教室的学生的集合,平面上与某一三角形相似的一切三角形的集合等等。
但应看到,Cantor的叙述不能当作集合概念的定义,因为,在这叙述中所谓“汇集”,“整体”,并不比“集合”简单,不过是它的同义语罢了。

在本书中,我们把“集合”和“属于”当作不定义的原始概念。
用$a \in A$记句子“a属于集合A”,用$a \notin A$记句子“a不属于集合A”。
在$a \in A$的情况下, 说a是集合A的元素,或说集合A包含a。

现在有个问题需要说明:集合的元素是什么?
我们的答复是,集合的元素还是集合。
这就是说,当我们写下$a \in A$时,其中的a和A都被认为是集合。
对读者来说,一个集合以另外一些集合为元素的情况是不生疏的。
例如,与一已知直线平行的一切直线组成一个集合,这集合的元素是直线,而直线又可看成点的集合。
读者可能感到不解的是,除了以集合为元素的集合外,为什么不考虑以其他事物为元素的集合?
我们的答复是:
1)我们讲的是抽象的集合论。
抽象集合论研究的只是集合和集合与集合之间的关系,不考虑集合以外的对象。
2)集合本来是不定义的概念,因而不能阻止我们把它们的元素仍看作是集合。
这样做并不妨碍把集合论的结构应用于其他分支(在允许应用的情况下)。

如上所述,“属于”关系(记为$\in$)是集合论中一个基本关系。
此外,我们把“等于”关系(记作$=$)看作是前于集合论而在集合论中沿袭下来的一个基本关系\footnote{有的作者利用“$\in$”定义“=”,并改动下面的外延公理。参看《Introduction to Axiomatic Set Theory》}。
集合A等于集合B(记作$A=B$)指的是A和B是同一集合,集合A不等于集合B(记作$A \neq B$)指的是A和B不是同一集合。
在$A = B$的情况下,可以理解为字母A和B是同一集合的不同记号,它们可以互相代替。
例如,如$a \in A$且$a = b$,则$b \in A$。

以下是关于“等于”和“属于”这两个基本关系的公理:

\fbox{
 \parbox{0.80\textwidth}{
 \textbf{外延公理}\\
对于集合A,B,如果对于任何的$x: x \in A$当且仅当$x \in B$,则$A=B$。
}
}

%\begin{spacing}{1.5}
%\end{spacing}

这就是说,如果两个集合有完全一样的元素,则它们是同一集合。

【注1】按照上面对于$A=B$的理解,外延公理的逆命题显然成立,即,如$A=B$,则对于任何的$x: x \in A$当且仅当$x \in B$。

此外,读者可能认为外延公理本身也是按照逻辑显然成立,不必列为公理。
难道元素完全一样的两个集合不是同一集合吗?
问题不是这样简单。
公理涉及到“=”和“$\in$”两个概念(“元素”概念是用“$\in$”概念解释的)。
如上所述,“=”概念算是明确了,但“$\in$”概念是不定义的。
在日常用语中,“属于”概念并不明确。
例如,用x记教室,用A记学生,从权利上看,不妨把学生A在教室x上课说成是x属于A。
如果两个学生A和B总在同样的教室上课(不论教室变到哪里),那么, 对于任何的教室x,x总是同时属于A和B的。
但A和B却不见得是同一个学生。
可见这样意义的“属于”就不符合外延公理。
外延公理实际是规定了集合论中“$\in$”的用法。

【注2】从外延公理和注1不难推出,集合$A \neq B$的必要且充分条件:存在某个x,$x \in A$但$x \notin B$,或者存在某个y,$y \in B$但$y \notin A$。

以下是我们的另一个公理:

\fbox{
\parbox{0.80\textwidth}{
\textbf{空集公理}\\
存在一个不包含任何元素的集合,这就是说,存在一个集合A,使得对于任何的x,$x \notin A$。
}
}

空集公理含有以下两层意思:
(1)我们承认了存在至少一个集合(当然不排斥存在其他集合),这样,我们的集合论就不是无的放矢了。
(2)我们承认了不含任何元素的“汇集”是一个集合,排除通常认为集合必须包含元素的想法。

【空集公理的注】空集公理中的集合A是唯一的。

这是因为,如果A和B都是这样的集合,即对于任何的x,同时有$x \notin A$和$x \notin B$,这等于说对于任何的x,同时有$x \in A$和$x \in B$。
按外延公理,$A=B$。

【定义】不含任何元素的集合叫空集,记以$\emptyset$。

\section{集合论的公式和条件}

现在,暂时回到直观的或经验的观点来议论集合。
我们认为一个集合是由一些确定的事物组成的一个集体(参看上节Cantor关于集合的描述)。
这就是说,任何一个事物属不属于这个集合应是完全能够判断的。
例如说,某教室内一切身高超过$1.80$米的学生组成的集合。
这个集合完全确定(当然,可能是空集)。
但如说,“某教室内一切高个子(假定不给标准)学生”,这就不能形成集合。
在第一个例子里,关于集合的元素的条件是明确的,而在第二个例子里条件却不明确。

在人们的经验里,往往认为用一句合乎语法的语言来表达某一集合的元素的“条件”是否明确,是容易判断的。
但请看以下悖论\footnote{我们已把悖论汉化}:

\fbox{
\parbox{0.90\textwidth}{
\textbf{Berry悖论}\\
在一切自然数中,考虑“能用少于二十四个汉字表达的自然数”。
}
}

例如:
\begin{itemize}[label=$\circ$]
\item 一千九百八十五(七个汉字)
\item 三百九十八与五的积(九个汉字)
\item 公元一千九百八十六年零时全世界的人口数(十九个汉字)
\end{itemize}

\noindent 它们所表示的自然数都符合“条件”。
现在看这里给出的“条件”是否明确?
我们知道,汉字不超过六万个,于是用少于二十四个汉字形成的字组不会超过$60000+60000^{2} +...+60000^{23}$个。
可见能用少于二十四个汉字表示的自然数只有有限多个。
但自然数是无穷尽的,故必存在自然数,不能用少于二十四个汉字表示。
设$n_{0}$是其中的最小数。于是$n_{0}$可说成是:
\begin{center}
“不能用少于二十四个汉字表示的自然数中最小的一个”。
\end{center}
\noindent 从纯形式上看,自然数$n_{0}$只用二十三个汉字就表示出来了,所以$n_{0}$符合“条件”。
但从语义上看,$n_{0}$是“不能用少于二十四个汉字表示的自然数”,所以$n_{0}$不符合“条件”!
由此可见,这里给出的貌似明确的“条件”归根到底是不明确的。

以上悖论说明,用通用语言表达的“条件”是否明确并非永远容易判断的。
应该对怎样才算“明确”给出标准。
但是,这样的标准对通常语言来说是难以给出的(这个难题留给语言学家)。
好在我们谈论的是数学,特别谈论的是集合论。
集合论只考虑集合和集合与集合之间的关系。
这样,我们可以把“条件”限制在一个明确的范围内。

先介绍“集合论的公式”这一概念。
在上节我们已经说过,“$=$”和“$\in$”是集合论中两个基本关系。
另外,集合论中还有七个逻辑关系,分别用以下七个符号来记(每个符号后面写的是它的原义):
\begin{itemize}[label=$\circ$]
\item $\neg$$\quad$——$\quad$非
\item $\lor$$\quad$——$\quad$或(非此即彼,包括亦此亦彼)
\item $\land$$\quad$——$\quad$且
\item $\Rightarrow$$\quad$——$\quad$蕴涵(亦蕴含?)
\item $\Leftrightarrow$$\quad$——$\quad$当且仅当
\item $\forall()$$\quad$——$\quad$对于任何(每个,一切)的(*),都有
\item $\exists()$$\quad$——$\quad$存在某个(*),使得
\end{itemize}

现在规定:
\begin{enumerate}[label=\arabic*)]
\item ($a=b$)和($a \in b$)(其中a,b可用任何字母代替)是公式,被称为原子公式。
\item 如$\varphi$,$\psi$是公式,则$(\neg \varphi)$,$(\varphi \lor \psi)$,$(\varphi \land \psi)$,$(\varphi \Rightarrow \psi)$,$(\varphi \Leftrightarrow \psi)$是公式。
\item 用$\varphi(x)$记含$x$的一个公式(还可含其他字母),则($\forall x \varphi(x)$)(对于任何的$x$,$\varphi(x)$成立)和($\exists x \varphi (x)$)(存在某个$x$,使得$\varphi (x)$成立)是公式。
\item 除1),2),3)所列者外,都不算公式。
\end{enumerate}

在不致引起混淆的情况下,可以略去公式中的一些括号。
例如,$x \in A$,$x \in B$和$A=B$都是原子公式,$x \in A \Leftrightarrow x \in B$和$\forall x (x \in B \Leftrightarrow x \in B)$都是公式,最后,
\begin{center}
$\forall A \forall B (\forall A (x \in A \Leftrightarrow x \in B) \Rightarrow A = B)$
\end{center}
还是公式,它就是外延公理的公式表达式。

又如,$x \in A$是原子公式,$\neg(x \in A)$是公式,$\forall x (\neg (x \in A))$也是公式,最后,
\begin{center}
$\exists A \forall (\neg (x \in A))$
\end{center}
还是公式,它就是空集公理的公式表达式。

在本书中,按严格意义说,当我们说到“公式”时,就指如上所述的集合论公式,这些公式都是从原子公式出发,按照上述1),2),3)的规定,运用限定的逻辑符号形成的。
以后,为了简化书写,也允许用另外的符号(例如用$a \notin b$代替$\neg (a \in b)$,用$a \neq b$代替$\neg (a=b)$等等)或用通常的语言来表达公式,但要求这样的表达语言时最终能够写成集合论的公式的。

\hspace*{\fill}

我们已经明确什么是集合论的公式了,现在定义“集合论的条件”:

当且仅当$C(x)$是含x的一个公式(还可含其他字母)并且不含$\forall x$及$\exists x$,我们说$C(x)$是$x$的一个集合论条件,简称条件。

例如,$x \neq x$是$x$的一个条件。
当然这样的$x$并不存在,即$\forall x (\neg(x \neq x))$。
另一方面,我们有$\forall x (\neg(x \in \emptyset))$。
因此,对于任何的$x$,$(x \neq x)$,$(x \in \emptyset)$同时不成立,这等于说它们同时成立。
于是,
\begin{center}
$\forall x (x \in \emptyset \Leftrightarrow x \neq x)$。
\end{center}
可以说,$x \neq x$是使$x$成为空集$\emptyset$的元素的一个条件。
一般来说:

对于一个已知条件$C(x)$,如果存在一个集合$A$,恰好包含使$C(x)$成立的一切$x$,即
\begin{center}
$\exists A \forall x (x \in A \Leftrightarrow C(x))$,
\end{center}
就说$C(x)$是使$x$成为集合$A$的元素的一个条件,并记
\begin{center}
$A={x|C(x)}$。
\end{center}

例如可写
\begin{center}
$\emptyset={x|x \neq x}$。
\end{center}

现在,我们终于明白明确了什么是集合论的条件,它们被限制在前述规定的范围内。
我们不考虑这个范围之外的“条件”。
例如,Berry悖论中的“条件”,除非证明它能够表示为集合论的条件,我们就不考虑它。
(你能用集合论的条件确切地表示“用...汉字表示的...”吗?)

另外,我们还明确了$C(x)$是使$x$成为集合$A$的元素的条件的意义,即$\forall x (x \in A \Leftrightarrow C(x))$。
必须注意,这是在集合A存在的情况下说话的。
例如,只是空集公理保证空集$\emptyset$的存在以后,我们才说$x \neq x$是使$x$成为$\emptyset$的元素的条件。
现在的问题是:任何一个集合论的条件$C(x)$都能确定一个集合吗?
下面的悖论给出否定的答复:

\fbox{
\parbox{0.90\textwidth}{
\textbf{Russell悖论}\\
看集合论的条件$x \notin x$,并假定满足这条件的一切$x$组成一个集合,即假定存在集合$A$,使
\begin{align}
\forall x (x \in A \Leftrightarrow x \notin x) \tag{1}
\end{align}
我们看$A \in A$还是$A \notin A$?如$A \in A$,即在(1)的左边以$A$代$x$,于是由(1)推出$A \notin A$。
如$A \notin A$,即在(1)的右边以$A$代$x$,于是由(1)推出$A \in A$。
可见如满足条件$x \notin x$的一切$x$组成一个集合$A$,那么,在任何情况下,都将导致$A \in A$与$A \notin A$同时成立的矛盾。
}
}

Russell悖论告诉我们,即使条件$C(x)$是明确的集合论条件,满足它的一切$x$也不一定形成一个集合。
下面对这悖论做些非正式的解释:一个集合$x$不是自己的元素应该是一个普遍的原则。
这样,条件$x \notin x$太广泛论,满足这条件的$x$太漫无边际了,以致如认为它们形成集合,就会导致逻辑上的矛盾。
这部分说明了我们考虑的集合不能是过于庞大的汇合物。

\section{子集}

【定义】集合A叫做集合B的子集,记为$A \subset B$,当且仅当A的元素都是B的元素,即当且仅当
\begin{center}
$\forall x (x \in A \Rightarrow x \in B)$。
\end{center}

此时说A包容于B,也说B包含A,并说B是A的包容集。

当且仅当A是B的子集又不与B重合,即当且仅当$A \subset B \land A \neq B$时,说A是B的真子集,记为$A \subsetneqq B$。

【注】包容关系$\subset$具有以下的性质:对于集合A,B,C:
\begin{enumerate}[label=\arabic*),itemindent=2em]
\item (自反性)$A \subset A$。
\item (反对称性)$A \subset B \land B \subset A \Rightarrow A = B$。
\item (传递性)$A \subset B \land B \subset C \Rightarrow A \subset C$。
\end{enumerate}

性质1),3)可直接由子集定义推出;性质2)是子集定义与外延公理结合的结果。
性质1)说的是一个集合总是总是自己的子集(非真子集)。
性质2)说的是如两集合互为子集,则它们重合。

【定理1】空集是每个集合的子集,即对于任何的集合$A$,$\emptyset \subset A$。

【证】只要证明$\forall x (x \in \emptyset \Rightarrow x \in A)$就可以了。
事实上,对于任何的$x$,$x \in \emptyset$总是错的。
所以对于任何的$x$,
$$x \in \emptyset \Rightarrow x \in A$$
就是对的\footnote{逻辑法则,参考$p \Rightarrow q$真值表}。

现在回到上节末尾的议论。
我们知道,满足一个已知条件$C(x)$的一切$x$不一定形成一个集合。
像Russell悖论中的“汇合物”${x \mid x \notin x}$就是过于庞大,以致不能当作一个集合来考虑。
但是,可以设想,既然集合论的条件是明确的,那么,如果我们不是漫无边际地考虑满足条件$C(x)$的一切$x$,而是在一个已知集合$A$的内部考虑满足$C(x)$的一切$x$,这些$x$总该形成一个集合了吧。
这就是以下的公理。

\fbox{
\parbox{0.90\textwidth}{
\textbf{子集公理}\\
对于每个条件$C(x)$,及任何集合$B$,存在一个集合$A$(字母$A$不出现在$C(x)$中),恰好包含$B$中一切满足$C(x)$的元素$x$,即,
\begin{center}
$\forall B \exists A \forall x (x \in A \Leftrightarrow x \in B \land C(x))$
\footnote{按严格意义,子集公理不是一条公理。对于每个条件$C(x)$,有一条“子集公理”。
这里列出的是无数多条公理的一个概括。
子集公理又称析取公理}。
\end{center}
}
}

这里,按照上节的规定,可写
\begin{center}
$A={x \mid x \in B \land C(x)}$。
\end{center}
为了强调$A$是$B$的子集,我们写
    \begin{center}
$A={x \in B \mid C(x)}$。
\end{center}

【子集公理的注】子集公理中的集合$A$是唯一的。

事实上,如果$A^{'}$也符合公理的要求,即
\begin{center}
$\forall x (x \in A^{'} \Leftrightarrow x \in B \land C(x))$,
\end{center}
则
\begin{center}
$\forall x (x \in A \Leftrightarrow x \in A^{'})$。
\end{center}
按外延公理,我们得到
\begin{center}
$A=A^{'}$。
\end{center}

由子集公理可以推出一个很有意义的结果:

【定理2】不存在“万有集”,即不存在包含一切集合的集合。

【证】只要证明,对于任何集合$B$,总存在一个集合,不是$B$的元素就可以了。
按子集公理,
\begin{center}
$C={x \in B \mid x \notin x}$
\end{center}
是一个集合,就是说,存在一个集合$C$,使得
\begin{align}
\forall x (x \in C \Leftrightarrow x \in B \land x \notin x)。 \tag{1}
\end{align}
以下证明$C \notin B$。
假如不然,即$C \in B$。
现在看$C \in C$还是$C \notin C$?
如$C \in C$,即在(1)的左边以$C$代$x$,于是由(1)推出$C \notin C$。
如$C \notin C$,则因已假定$C \in B$,故在(1)的右边以$C$代$x$成立,于是由(1)推出$C \notin C$。
由此可见$C \in B$永远导致矛盾。
故$C \notin B$

从定理的证明可以看出,定理2实际是Russell悖论的变种。
在集合论尚处于“朴素的”阶段时,由于集合概念没有明确的公理限制,人们很难拒绝Russell悖论中所说的$A$是一个集合,从而形成了一个悖论。
但是,在集合论公理化以后,按照Russell悖论的精神,从子集公理出发,就能推出一个合乎逻辑的结果:
对于我们考虑的集合概念来说,不存在包含一切集合的“万有集”。

【推论】不存在集合,包含一切满足条件$x=x$的集合$x$。

事实上,$x=x$对于任何集合$x$都成立。
说一个集合包含一切满足$x=x$的集合$x$,等于说这集合包含一切集合。

把这结果同空集概念比较一下是有益的。
条件$x=x$对于任何$x$都成立,但以一切这样的$x$为元素的集合不存在。
另一方面,使条件$x \neq x$的$x$不存在,但$\emptyset={x \mid x \neq x}$却是我们已经承认存在的空集!
读者切勿把满足某条件的集合不存在同某集合(空集)的元素不存在混为一谈。

\section{偶集}

现在看一下问题:给定集合$a$,是否存在集合,恰好以$a$为元素?
就是说,是否存在集合$A$,使
\begin{center}
$\forall x (x \in A \Leftrightarrow x=a)$?
\end{center}
类似的问题:给定集合$a$,$b$,是否存在集合,恰好以a和b为元素?
就是说,是否存在集合A,使
\begin{center}
$\forall x (x \in A \Leftrightarrow x=a \lor x=b)$?
\end{center}
前面设立的公理都不能保证这样集合的存在。
需要设立新的公理。
为了简便,我们只设立一条公理,保证以上两种集合中的后一种集合存在,并把前一种集合当作特殊情况。

\fbox{
 \parbox{0.80\textwidth}{
 \textbf{偶集公理}\\
给定集合$a$,$b$,则存在一个集合$A$,恰好以$a$,$b$为元素,即
\begin{center}
$\forall a \forall b \exists A \forall x (x \in A \Leftrightarrow x=a \lor x=b)$。
\end{center}
}
}

【偶集公理的注】偶集公理中的集合$A$是唯一的。

证明与子集公理的注的证明类似。

【定义】对于集合$a$,$b$,记
\begin{center}
$\{a,b\}=\{x \mid x=a \lor x=b\}$。
\end{center}
并称之为以$a$,$b$为元素的偶集。

【注】${a,b}={b,a}$。

这是因为,
\begin{center}
$x=a \lor x=b \Leftrightarrow x=b \lor x=a$
\footnote{对于命题p,q,我们有:$p \lor q \Leftrightarrow q \lor p$;$p \land q \Leftrightarrow q \land p$}。
\end{center}
在这个意义下,偶集又被称为无序偶。

在偶集定义中,并未限定$a \neq b$。
以下是$a=b$的情况:

【定义】记$\{a\}=\{a,a\}$,并称之为以$a$为元素的单集。

显然,单集定义还可写成
\begin{center}
$\{a\}=\{x \mid x=a\}$。
\end{center}
这就是说,单集${a}$是以$a$为唯一元素的集合。

在本节以前,对于具体的集合,我们只承认了空集$\emptyset$的存在。
在设立偶集公理以后,我们可以无休止地构造另外的具体集合了。
例如,可以构造单集$\{\emptyset\}$,$\{\{\emptyset\}\}$,$\{\{\emptyset, \{\emptyset\}\}\}$,等等;
可以构造偶集$\{\emptyset, \{\emptyset\}\}$,$\{\{\emptyset\},\{\emptyset, \{\emptyset\}\}\}$,等等。
这些都是受到承认的集合。

最后,应提醒读者注意区分集合$a$与单集$\{a\{$:它们之间的关系是前者是后者的唯一元素。
例如,$\emptyset$是不含元素的空集,而$\{\emptyset\}$是以$\emptyset$为唯一元素的集合。
$\{\{\emptyset\}\}$是以$\{\emptyset\}$为唯一元素的集合。
做个通俗的比喻:如把$a$看作一个田径运动员,那么,$\{a\}$就可以看作以$a$为唯一队员的田径队。
在运动会上,个人成绩和团体成绩要分别计算的。

\section{并集}

设给定集组\footnote{我们说“集组”,指的是“以集合为元素的集合”。“集组”就是集合,这里采用“集组”这一名称只是为了分清层次。}$M$。
把$M$包含的一切集合合并起来,就是说,把这些集合的元素统统汇集起来,其结果能否形成集合?
以下公理做出肯定的允诺。

\fbox{
 \parbox{0.80\textwidth}{
 \textbf{并集公理}\\
对于一个集组$M$,存在一个集合$A$,恰好包含属于$M$中至少一个集合的一切元素,即
\begin{center}
$\forall M \exists A \forall x (x \in A \Leftrightarrow \exists X (X \in M \land x \in X))$。
\end{center}
}
}

【并集公理的注】有外延公理可知,并集公理中的集合$A$是唯一的。

为了简化书写,可把公理中关于$x$的条件

\section{交集}

\section{差集}

\section{幂集}
